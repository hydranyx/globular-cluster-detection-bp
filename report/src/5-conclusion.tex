\chapter{\label{chap:conclusion}Conclusion}

For \blobdog{} with a $B_{\text{threshold}} = 0.2$, 7 out of the 12 rasters containing known GCS are maintained in Area 1, the 1 containing a known GC is maintained in Area 2, and 16 out of 17 are maintained in Area 3. It is evident from these results that \blobdog{} with this $B_{\text{threshold}}$ does not work perfectly as a pre-processing method, as it filters out known GCs. However, it does keep 80\% (24 out of 30) of the known GCs. Furthermore, no raster filtered away by \blobdog{} contained any known GCs as identified by the clustering phase that follows. A total of 21 rasters which would have been identified by the clustering as containing a stellar structure were filtered away by \blobdog{}, 7 of which are in Area 1 and 14 in Area 3.

From Table~\ref{tb:cluster-removals} the stellar structures that are removed by \blobdog{} are predominantly galaxies (10 of 24) or do not correspond to any identifiable stellar structure (11 of 24). These clusters that have no known stellar structures, are either a failure of the Ant Colony algorithm or represent potentially undiscovered GC candidates. It is possible that the $B_{\text{threshold}}$ may be too large (especially as it discards known GCs) and thus some of these clusters that were filtered away may be on the cusp for \blobdog{}. However, lowering the $B_{\text{threshold}}$ to 0.1 results in \text{no} rasters being filtered away. Thus, the ideal $B_{\text{threshold}}$ lies between 0.1 -- 0.2. To summarize \blobdog{} operates as an effective exclusion criteria but it is not yet perfect in maintaining all known GCs.

For the Ant Colony algorithm, the amount of stars per raster influence the $\mu_{\text{pheromone}}$ as well as the distribution of stars across that raster. These quantities also influence the percentage of the visited stars in that raster, which can be observed from Table~\ref{tb:clusters-ant-stats}, when comparing Area 2: $\SI{2.0}{\degree}\times\SI{2.0}{\degree}$ versus Area 2: $\SI{4.0}{\degree}\times\SI{4.0}{\degree}$. The $\mu_{\text{pheromone}}$ values, the distribution of the pheromone values, and the stars' visitation percentage in Area 1 and Area 3 are quite similar despite the areas having very different characteristics. Area 1 has the lowest density of stars across a region while Area 3 features regions with an extremely high density of stars. Exploring the behavior of the Ant Colony algorithm with respect to the total number of stars and the density of the stars with the rasters is pivotal to optimizing the clustering phase that follows.

From Table \ref{tb:clusters-ant-stats} and Figures~\ref{fig:pheromone-run-1} to \ref{fig:pheromone-run-5} it can be observed that the ants behave similarly across each of the experiments. However, it becomes clear from the results of the Clustering that even the small variations in the ants behavior can result in distinct clusterings. This is evident in Table~\ref{tb:clusters-detected-areas} where the results of visitation percentage and pheromone distribution were similar but the number of clusters found in each experiment per area were different. Additionally the sporadic results from the clustering of Area 2 reveals that it is necessary to run the algorithm many times to identify the different clusters present and to gauge the certainty with which the ants discover it.

All areas find clusters except for Area 2: $\SI{4.0}{\degree}\times\SI{4.0}{\degree}$, which highlights the relationship between the functioning of the Ant Colony algorithm and the amount of stars contained within the rasters. Table~\ref{tb:clusters-detected-areas} shows that some rasters contain multiple clusters. The known GCs found in Area 1 and Area 2 are the same as found by \blobdog{} but Area 3 only finds 5 out of 17 known GCs. These results show promise but highlight the need to fine tune the control parameters of the Ant Colony and the clustering. In Area 1 and Area 3 many clusters were detected in rasters corresponding to the darker regions of the areas which were subsequently filtered by \blobdog{}. This shows that the Ant Colony may have a predisposition to identifying clusters more easily in regions with less stars.

Across the full pipeline, the system is able to identify 13 GCs out of the 30 known GCs and identifies a total of 13 rasters out of the 27 rasters containing known GCs. From Table~\ref{tb:results-aggregated}, 27 out of the 41 clusters that were identified by the pipeline corresponded to a known stellar structure. 13 of these corresponded to GCs and 14 corresponded to some other pre-existing stellar structure.

While the pipeline does not identify the majority of the GCs that exist, for those that it is able to identify, it can pinpoint their locations quite accurately. Further research in tuning the parameters and steering the behavior of the ants is expected to expand the number of GCs identified by the pipeline and to solidify the Ant Colony as a useful tool for exploring the Universe.
